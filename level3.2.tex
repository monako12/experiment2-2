\subsection{Level3.2: パラメータと収束能力の関連性について}
\subsubsection{関係性を確認するためのアプローチ}
3つのパラメータがどのような関係にあるかを検証するため、
ケース1,2,,,を設定し、学習曲線からその関係性について考察する。

\subsubsection{結果}
\begin{verbatim}
-------------------------------
10000,hidden=10,alpha=0.2,eta=0.1
10000,hidden=10,alpha=0.1,eta=0.1
9996.7,hidden=10,alpha=0.6,eta=0.1
9919.8,hidden=10,alpha=0.3,eta=0.1
9879,hidden=10,alpha=0.5,eta=0.1
9856.8,hidden=10,alpha=0.4,eta=0.1
9671.3,hidden=10,alpha=0.1,eta=0.2
9670.8,hidden=90,alpha=0.8,eta=1.7
9660.6,hidden=10,alpha=0.2,eta=0.2
9424.7,hidden=10,alpha=0.3,eta=0.2
------------------------------------
109,hidden=40,alpha=0.4,eta=1.4
108.4,hidden=30,alpha=0.5,eta=1.5
108.1,hidden=40,alpha=0.5,eta=1.4
106.9,hidden=30,alpha=0.5,eta=1.8
106.2,hidden=30,alpha=0.6,eta=1.8
105.6,hidden=30,alpha=0.6,eta=1.5
105.6,hidden=30,alpha=0.5,eta=1.7
105.3,hidden=30,alpha=0.6,eta=1.6
104.9,hidden=30,alpha=0.5,eta=1.6
101.3,hidden=30,alpha=0.6,eta=1.7
\end{verbatim}
上記の実行結果はスクリプトでの出力をsortコマンドでソートし,結果が悪いも
のと良いものを抜き出してきたものである.
左からseed値1000|~|10000までの収束した回数の平均値,hiddenの値,alphaの
値,etaの値となっている.
\subsubsection{考察}
悪いところの共通点はhiddenの値が10か90のどちらかになっている.結果が良かっ
た方を見るとhiddenは30か40を取っているのでhiddenの値は極端に大きい物や
小さい物は良くない.また,alpha値は結果が良い方を見ると,0.5,0.6,0.4に分けられ
る.この事からalphaの最適な値は0.5$\pm$1だという事が分かる.etaの値は大き
い方が良いが結果の悪い方の,下から三行目の結果からhidden値とetaが両方とも大きいと結果が悪くなる事が分かる.
