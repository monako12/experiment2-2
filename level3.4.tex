\subsection{Level3.4: 認識率を高める工夫}
%下図のように入力されたデータが想定していた入力と比べてサイズが異なったり、
%位置がずれている等、数字の一部が欠けている以外にも多様な要因による
%データ(情報)の劣化が考えられる。
%認識率を高めるにはどのような点を工夫すれば良いか?
%どのような方法でも構わないので、検討せよ。
\subsubsection{対象とする問題点}
以下に対象とする問題点を列挙する.
\begin{itemize}
\item 与えられた数字のサイズが相対的に小さい
\item 与えられた数字のサイズが相対的に大きい
\item 与えられた数字の位置がずれている
\item 与えられた数字の周りにノイズが入っている
\end{itemize}

\subsubsection{改善方法の提案}
\begin{enumerate}
\item 数字のサイズが相対的に大きいもしくは小さいなどの場合は,
与えられた数字に対して拡大・縮小を行い,%与えられた数字を
学習用の数字に近づけた上で認識させる方法がある.
\item 数字の位置がずれている場合も同様に,
与えられた数字を中央に移動させたものを認識させる方法が考えられる.
\item 数字の周りにノイズが入っている場合は,
学習用の数字と与えられた数字とを重ね合わせてANDをとったものを
認識させる方法がある.
\end{enumerate}
% 平均化した後の値が大きい部分だけに注目すれば,
% ノイズが軽減された数字が浮かび上がり認識率が

\subsubsection{考察}
与えられた数字に対して拡大・縮小を行う方法の問題点として,
どの程度の拡大および縮小を行えばよいのかが不明な点である.
少しずつ拡大・縮小の度合いを大きくしていき,それを認識させる
ことを繰り返すという方法が考えられるが,
これはやや非効率なやり方である. \\
 数字を中央に移動させる方法は,
数字の縦横の比率から中央となる場所を割り出し,数字を移動させる
ため,比較的単純な処理となる. \\
%%  続いて,与えられた数字を中央に移動させる方法であるが,
%% これは与えられた数字の縦横の比率から中央を割り出し,そこ
%% 比較的簡単な作業となる. \\
% 重ね合わせて平均化する方法では,
 ANDをとる方法では,ANDをとることで与えられた数字と学習用の数字の
共通部分のみを抽出できるため,
%平均化した後に値が大きい部分だけに注目することで,
%数字の周りのノイズが軽減され,相対的に数字をくっきり
%認識できるようになるため,
認識率を格段に向上させることができる.
しかし,与えた数字と学習用の数字のどちらかが相対的に大きかったり,
互いの位置がずれていたりすると,この手法はあまり効果を発揮しない.
そのため,改善方法の提案で述べた1と2の方法と組み合わせることで,認識率を高めることができると考える.
